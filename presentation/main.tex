\documentclass[xcolor=dvipsnames]{beamer}
\mode<presentation>

\usepackage[utf8]{inputenc}
\usepackage{polski}
\usepackage{mathptmx}

\setbeamertemplate{navigation symbols}{}
\setbeamertemplate{theorems}[numbered]
\setbeamertemplate{items}[default]
\setbeamercovered{dynamic}
\setlength{\unitlength}{1cm}
\setbeamertemplate{footline}[frame number]
\usepackage{setspace}
\usepackage{amsmath,amssymb}
\usepackage{amsthm}
\usepackage{pgf,pgfarrows}


\usepackage{graphics}
\usepackage{tikz}
\usepackage[plain]{algorithm}
\usepackage[noend]{algpseudocode}
\usepackage{scalefnt}
\usepackage{array}
\usepackage{colortbl}
\usepackage{svg}
\usepackage{float}

\usepackage{sansmathaccent}
\pdfmapfile{+sansmathaccent.map}


\usetikzlibrary{arrows,automata,snakes}
\usetheme{Frankfurt}
\usecolortheme{seahorse} %crane
\useinnertheme{circles}

\newtheorem*{lemat}{Lemat}
\newtheorem{twierdzenie}{Twierdzenie}
\newcommand{\myitem}{\item[$\vartriangleright$]}
\newcommand{\nota}[1]{{\color{gray} \emph{#1}}}
\newcommand{\HarmonicN}[1]{H_{#1}}
\newcommand{\BALL}[2]{\mathbf{B}(#1,#2)}
\newcommand{\DISC}[2]{\mathbf{D}(#1,#2)}
\newcommand{\SPHERE}[2]{\mathbf{S}(#1,#2)}
\newcommand{\BigO}[1]{\mathcal{O}\left(#1\right)}
\newcommand{\BigTh}[1]{\Theta\left(#1\right)}
\newcommand{\slfrac}[2]{\left.#1\middle/#2\right.}
\newcommand{\PR}[1]{\mathrm{Pr}\left[#1\right]}
\newcommand{\E}[1]{\mathbb{E}\left[#1\right]}
\newcommand{\var}[1]{\mathbb{V}\mathrm{ar}\left[#1\right]}
\newcolumntype{C}{>{\centering\arraybackslash}p{0.5cm}}
\newcolumntype{Y}{>{\columncolor{blue!5}}C}

\newcommand{\source}[1]{\caption*{Source: {#1}} }

\newcommand{\COMMAND}[1]{\textbf{#1}}

\newcommand{\CALL}[2]{\textbf{#1(}#2\textbf{)}}

\newcommand{\FUNC}[2]{\text{#1(}#2\text{)}}

\newcommand{\VAR}[1]{\textit{#1}}

\newenvironment{conditions}
{\par\vspace{\abovedisplayskip}\noindent\begin{tabular}{>{$}l<{$} @{${}-{}$} l}}
	{\end{tabular}\par\vspace{\belowdisplayskip}}

%\pgfpagesuselayout{4 on 1 with notes}[a4paper,border shrink=3mm]

%\setbeamercolor*{palette tertiary}{bg=Green}
%\setbeamercolor{frametitle}{fg=Green,bg=Green!10}
%\setbeamercolor{section in head/foot}{bg=Green!10}

\usepackage[nottoc]{tocbibind}

\title{Analiza eksperymentalna nowych algorytmów sortowania w miejscu}

\author{
	\textbf{Damian Baliński}
	\newline \newline
	Praca napisana pod kierunkiem 
	\newline
	\textbf{dra Zbigniewa Gołębiewskiego}
}


\date{2021, Wrocław}

\begin{document}

\begin{frame}[plain]{}
	\titlepage
\end{frame}

\section{Metodologia}
\begin{frame}[squeeze]{Motywacja}

	\newtheorem*{sortimporove*}{Cele tworzenia algorytmów hybrydowych}
	\begin{sortimporove*}
		\begin{itemize}
			\item poprawa wydajności
			\item optymalizacja przypadku pesymistycznego
			\item działanie w miejscu
			\item minimalizacja liczby porównań
		\end{itemize}
	\end{sortimporove*}

\end{frame}
\begin{frame}[squeeze]{Metodologia}

	\newtheorem*{methodology*}{Łączny koszt operacji}
	\begin{methodology*}
		$$C = \alpha n_c + 3n_s + n_a $$
	\end{methodology*}

	\begin{conditions}
		\alpha	&  wartość współczynnika kosztu 	\\
		n_c		&  liczba operacji porównania 		\\   
		n_s		&  liczba operacji zamiany miejsc 	\\
		n_a		&  liczba operacji przypisania		\\
	\end{conditions}

\end{frame}

\section{Podstawowe algorytmy sortujące}
\begin{frame}[squeeze]{Quick Sort}
	
	\newtheorem*{quicksort*}{Wady algorytmu Quick Sort}
	\begin{quicksort*}
		\begin{itemize}
			\item $O(n^2)$ dla przypadku pesymistycznego
			\item duża liczba porównań
		\end{itemize}
	\end{quicksort*}
	
\end{frame}
\begin{frame}[squeeze]{Wady algorytmu Merge Sort}
	
	\begin{itemize}
		\item Konieczność posiadania dodatkowej pamięci o wielkości $O(n)$
		\item Dodatkowy nakład czasowy związany z alokacją oraz zwalnianiem pamięci
	\end{itemize}

\end{frame}


\section{Hybrydowe algorytmy sortujące}
\begin{frame}[squeeze]{QuickMerge Sort}
	
	\begin{figure}[]
		\makebox[\textwidth][c]{\includesvg[inkscapelatex=false,width=1.0\columnwidth]
			{figures/quick-merge-sort-schema.svg}}
	\end{figure}
	
\end{frame}

\begin{frame}[squeeze]{QuickMerge Sort - pseudokod}
	
	\begin{algorithm}[H]
		\unboldmath
		\thinmuskip=6mu
		\begin{algorithmic}[1]
			\Procedure{QuickMergeSort(arr)}{}
			
			\State
			\If {$\FUNC{len}{\VAR{arr}} = 1$} \COMMAND{end} \Comment{warunek wyjścia}
			\EndIf
			
			\State
			\State $\VAR{arr}_1, \VAR{arr}_2 \gets \CALL{Partition}{\VAR{arr}}$ \Comment{partycjonowanie}
			
			\State
			\If {$\FUNC{len}{\VAR{arr}_1} < \FUNC{len}{\VAR{arr}_2}$} \Comment{sortowanie}
			\State $\VAR{buffer} \gets \VAR{arr}_2$
			\State $\CALL{MergeSortBySwaps}{\VAR{arr}_1, \VAR{buffer}}$
			\State $\CALL{QuickMergeSort}{\VAR{arr}_2}$
			\Else
			\State $\VAR{buffer} \gets \VAR{arr}_1$
			\State $\CALL{MergeSortBySwaps}{\VAR{arr}_2, \VAR{buffer}}$
			\State $\CALL{QuickMergeSort}{\VAR{arr}_1}$
			\EndIf

			\EndProcedure
		\end{algorithmic}
		\boldmath
		\thinmuskip=3mu
	\end{algorithm}
	
\end{frame}
\begin{frame}[squeeze]{QuickMerge Sort - pseudokod}
	
	\begin{algorithm}[H]
		\unboldmath
		\thinmuskip=6mu
		\begin{algorithmic}[1]
			\Procedure{QuickMergeSort(arr)}{}
			
			\State
			\If {$\FUNC{len}{\VAR{arr}} = 1$} \COMMAND{end} \Comment{warunek wyjścia}
			\EndIf
			
			\State
			\State $\VAR{arr}_1, \VAR{arr}_2 \gets \CALL{Partition}{\VAR{arr}}$ \Comment{partycjonowanie}
			
			\State
			\If {$\FUNC{len}{\VAR{arr}_1} < \FUNC{len}{\VAR{arr}_2}$} \Comment{sortowanie}
			\State $\VAR{buffer} \gets \VAR{arr}_2$
			\State $\CALL{MergeSortBySwaps}{\VAR{arr}_1, \VAR{buffer}}$
			\State $\CALL{QuickMergeSort}{\VAR{arr}_2}$
			\Else
			\State $\VAR{buffer} \gets \VAR{arr}_1$
			\State $\CALL{MergeSortBySwaps}{\VAR{arr}_2, \VAR{buffer}}$
			\State $\CALL{QuickMergeSort}{\VAR{arr}_1}$
			\EndIf

			\EndProcedure
		\end{algorithmic}
		\boldmath
		\thinmuskip=3mu
	\end{algorithm}
	
\end{frame}
\begin{frame}[squeeze]{Rodzina algorytmów QuickMerge Sort}
	
	\begin{figure}[]
	\makebox[\textwidth][c]{\includesvg[inkscapelatex=false,width=1.15\columnwidth]
		{figures/quick-merge-sort-deterministic-nondeterministic.svg}}
	\end{figure}

\end{frame}
\begin{frame}[squeeze]{Intro Sort - pseudokod}
	
	\begin{algorithm}[H]
		\unboldmath
		\thinmuskip=6mu
		\begin{algorithmic}[1]
			\Procedure{IntroSort(arr, depth)}{}
			
			\State
			\If {$\FUNC{len}{\VAR{arr}} \leq \VAR{maxLength}$} \Comment{sortowanie przez wstawianie} \label{line:insertion-sort}
			\State $\CALL{InsertionSort}{\VAR{arr}}$
			
			\State
			\ElsIf {$\VAR{depth} = 0$} \Comment{sortowanie przez kopcowanie} \label{line:heap-sort}
			\State $\CALL{HeapSort}{\VAR{arr}}$
			
			\State
			\Else \Comment{szybkie sortowanie} \label{line:quick-sort}
			\State $\VAR{arr}_1, \VAR{arr}_2 \gets \CALL{Partition}{\VAR{arr}}$
			\State $\CALL{IntroSort}{\VAR{arr}_1, \VAR{depth-1}}$
			\State $\CALL{IntroSort}{\VAR{arr}_2, \VAR{depth-1}}$
			\EndIf

			\EndProcedure
		\end{algorithmic}
		\boldmath
		\thinmuskip=3mu
	\end{algorithm}
	
\end{frame}

\begin{frame}[squeeze]{Rodzina algorytmów Intro Sort}
	
	\begin{figure}[]
	\makebox[\textwidth][c]{\includesvg[inkscapelatex=false,width=1.15\columnwidth]
		{figures/intro-sort-deterministic-nondeterministic.svg}}
	\end{figure}

\end{frame}

\end{document} 