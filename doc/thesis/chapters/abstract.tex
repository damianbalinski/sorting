\chapter{Wstęp}
\thispagestyle{chapterBeginStyle}

Ogólna historia algorytmów sortujących. Motywacja tworzenie algorytmów 

\iffalse
- początki sortwania (Bubble Sort, Insertion Sort)
- przełom 1 (Quick Sort - brytyjski naukowiec Antony Hoare)
- przełom 2 (Vladimir Yaroslavskiy - )
- dalsze wynalazki (QuickMerge, Intro Sort)

Problem porządkowania zbioru elementów towarzyszy informatyce od początku.
Nic więc dziwnego, że sortowanie jest jednym z najbardziej powszechnych problemów
współczenej informatyki. 

W literaturze idea sortowania danych po raz pierwszy pojawiła się w ..., kiedy to ...
opublikował pracę ...,

Od tamtego czasu wiele rzeczy się zmieniło, powstawały coraz to nowsze algorytmy sortujące.

Prawdziwym przełomem był rok ... kiedy to ... opracował algorytm Dual Pivot Quick Sort.
Przez kilka następnych lat koncepcja sortowania wielopiwotowego została niesamowicie
wyeksploatowana, a efekty badań z tamtego okresu widoczne są we współczesnych implementacjach.
Np. biblioteka standardowa javy do dnia dzisiajszego (java ...) stsuje algorytm
dual pivot quick sort jako główny algorytm sortujący.

Mijały lata, a wiemy że jeżeli coś się nie rozwija to w świecie IT cofa się.
Wiemy że każdy z algorytmów ma swoje słabe i mocne strony. Quick sort ... merge sort ...

Ludzie zastanawiali się w jaki sposób połączyć algorytmy sortujące w taki sposób, aby
kożystać z tego, co mają najlepsze: tak narodził się pomysł na QuickMergeSort

Powstały algorytmy hybrydowe, czyli połączenie wielu podstawowych algorytmów sortujących
w specyficzny sposób. Dzięki czemu możliwe jest wykorzystywanie zalet, które niesie ze
sobą każdy z algorytów oraz niwelowanie wad. Algorytmy hybrydowe łączą ze sobą najlepsze
części algorytmów podstawowych.
\fi

\section{Cel pracy}
Motywem przewodnim pracy jest analiza nowoczesnych algorytmów sortujących w miejscu, takich jak koncepcja QuickMerge Sort. W tym celu przygotowano analizę porównawczą podstawowych algorytmów sortujących oraz dokonano przeglądu
zmodyfikowanych wersji tych algorytmów oraz przeanalizowano nowoczesne algorytmy hybrydowe, będące połączeniem dwóch
lub wielu algorytmów podstawowych.

% TODO jak zdaze to tez proba modyfikacji istniejacych algorytmow

\section{Zakres pracy}
Aby ułatwić analizę algorytmów przygotowany został silnik testujący oraz silnik graficzny, które w oparciu o plik
konfiguracyjny przeprowadzają testy oraz tworzą wizualizację wyników tych testów. Wykorzystując podane narzędzia
została przeprowadzona analiza podstawowych oraz hybrydowych algorytmów.

\section{Przegląd literatury}
Ogólny opis pracy Sebastiana Wilda. Pomysł na algorytm QuickMerge Sort.

\section{Zawartości pracy}
Ogólny sposób organizacji dokumentu.
Rozdział pierwszy - analiza matematyczna problemu.
Rozdział drugi - przegląd podstawowych algorytmów sortujących.
Rozdział trzeci - przegląd hybrydowych algorytmów sortujących.