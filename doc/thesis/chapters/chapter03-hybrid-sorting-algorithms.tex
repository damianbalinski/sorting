\chapter{Przegląd hybrydowych algorytmów sortujących}
\thispagestyle{chapterBeginStyle}

\section{Główne sposoby modyfikacji algorytmów}
1. Podmiana algorytmów składowych, np. inny algorytm partycjonowania.\\
2. Łączenie wielu algorytmów w jeden.

\section{Quick Sort z różnymi algorytmami partycjonowania}
Wykresy porównujące wydajność algorytmów rodziny quick sort w zależności od wyboru algorytmu partycjonowania:
partycjonowanie zwykłe, median of three, median of medians.


\section{QuickMerge Sort}
Koncepcja algorytmu, mocne strony (Merge Sort bez konieczności alokacji pamięci)

\subsection{Pseudokod}

\subsection{Analiza algorytmu}
Wykresy liczby wykonywanych operacji w porównaniu do algorytmów bazowych.
Wykresy gęstości liczby wykonywanych operacji dla stałej liczby n, np n = 10000.

\subsection{Wnioski}
Wyniki analizy porównawczej


\section{Intro Sort}
Ogólny opis algorytmu, gdzie jest wykorzystywany (std::sort w g++), zalety.

\subsection{Pseudokod}

\subsection{Analiza algorytmu}
Wykresy liczby wykonywanych operacji w porównaniu do algorytmów bazowych.
Wykresy gęstości liczby wykonywanych operacji dla stałej liczby n, np n = 10000.

\subsection{Wnioski}
Wyniki analizy porównawczej

% TODO jak zdaze proba modyfikacji algorytmu intro sort

% TODO jak zdaze dodac Quick Heap Sort

% TODO merge sort na wiele czesci
