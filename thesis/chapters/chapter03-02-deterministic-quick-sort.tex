\section{Rodzina deterministycznych algorytmów Quick Sort}
Podczas analizy algorytmów sortujących z rodziny Quick Sort zostały przetestowane wariacje algorytmów z różnymi metodami partycjonowania oraz różnymi deterministycznymi metodami wyboru piwota. Do partycjonowania danych wejściowych wykorzystano poniższe metody:

\begin{itemize}
	\setlength\itemsep{0em}
	\item \BOLD{metoda Lomuto} - jest domyślnym algorytmem partycjonowania w projektowanym systemie. Tablica jest iterowana od pierwszego do ostatniego elementu. Elementy mniejsze od piwota są przenoszone na lewą część tablicy, zaś elementy większe od piwota na jej prawą część. Algorytm kończy się w momencie przeniesienia ostatniego elementu.
	\item \BOLD{metoda Hoare} - tablica jest partycjonowana za pomocą dwóch iteratorów umieszczonych po przeciwnych stronach tablicy oraz skierowanych do jej środka. Pojedyncza iteracja trwa do momentu napotkania dwóch elementów, które nie znajdują się w odpowiednich częściach tablicy, tzn. element po lewej stronie jest większy od piwota, oraz element po prawej stronie jest mniejszy od piwota. Wtedy elementy znajdujące się w miejscu iteratorów są zamieniane miejscami oraz algorytm jest kontynuowany. Program kończy się w momencie spotkania obydwu iteratorów.
\end{itemize}

Wykonując testy brano pod uwagę następujące metody wyboru piwota:
\begin{itemize}
	\setlength\itemsep{0em}
	\item \BOLD{ostatni element} - piwotem jest ostatni element tablicy,
	\item \BOLD{mediana z trzech} - piwotem jest mediana z pierwszego, środkowego oraz ostatniego elementu tablicy,
	\item \BOLD{pseudo-mediana z dziewięciu} - piwotem jest mediana z dziewięciu równo oddalonych od siebie elementów, z których pierwszym jest pierwszy element tablicy, oraz ostatnim jest ostatni element tablicy,
	\item \BOLD{mediana-median z pięciu} - piwot jest medianą pięciu median obliczanych rekurencyjnie,
	\item \BOLD{mediana-median z trzech} - piwot jest medianą trzech median obliczanych rekurencyjnie.\\
\end{itemize}

\subsection{Analiza porównawcza algorytmów}

Analizując wydajność algorytmów z rodziny Quick Sort, badano liczbę wykonywanych operacji atomowych z podziałem na operacje porównania, zamiany miejsc oraz przypisania. Dodatkowo badano łączny koszt wykonanych operacji jako sumę ważoną liczby operacji atomowych, z uwzględnieniem wartości współczynnika kosztu. Badając łączną liczbę wykonanych operacji założono stałą wartość współczynnika kosztu $\alpha = 1.0$.\\

Dla losowych danych wejściowych (\ref{fig:quick-sort-deterministic-pivot-random}), czyli przypadku średniego w klasycznym algorytmie Quick Sort, przy założeniu że operacja porównania jest czasowo równoważna operacji przypisania ($\alpha=1.0$), partycjonowanie metodą Hoare jest wydajniejsze lub tak samo wydajne jak partycjonowanie metodą Lomuto. Najlepsze wyniki otrzymano poprzez połączenie partycjonowania metodą Hoare z wyborem piwota jako ostatni element tablicy. Najgorsze wyniki otrzymano wybierając piwot jako mediana-median z pięciu, przy czym najgorszy wynik osiągnięto niezależnie od wyboru metody partycjonowania.\\

Dla danych wejściowych posortowanych w odwrotnej kolejności (\ref{fig:quick-sort-deterministic-pivot-reversed}), czyli dla przypadku pesymistycznego w klasycznym algorytmie Quick Sort, przy założeniu że operacja porównania jest czasowo równoważna operacji przypisania ($\alpha=1.0$), partycjonowanie metodą Hoare jest również wydajniejsze lub tak samo wydajne jak partycjonowanie metodą Lomuto. Najlepsze wyniki otrzymano poprzez połączenie partycjonowania metodą Hoare z wyborem piwota jako mediana z trzech. Najgorsze wyniki otrzymano dla klasycznego algorytmu Quick Sort, czyli poprzez połączenia partycjonowania metodą Lomuto z wyborem piwota jako ostatni element tablicy.\\

Analizując łączny koszt wykonanych operacji (\ref{fig:quick-sort-deterministic-pivot-cost-factor}) w zależności od wartości współczynnika kosztu $\alpha$ można zauważyć, że dla typów danych o współczynniku kosztu większym bądź równym wartości $\alpha = 2.0$, partycjonowanie metodą Lomuto jest wydajniejsze niż partycjonowanie metodą Hoare. Wyniki te można interpretować jako sortowanie struktur złożonych z co najmniej dwóch typów prostych. Najskuteczniejszą z badanych strategii sortowania jest partycjonowanie metodą Lomuto z wyborem piwota jako mediana z trzech elementów. Metoda ta jest najskuteczniejsza powyżej wartości $\alpha = 2.0$. Najmniejszą skuteczność ma wybór piwota jako mediana-median z pięciu elementów, przy czym wynik najgorszy osiągnięto dla obydwu metod partycjonowania.\\

Porównując liczbę wykonanych operacji pomiędzy badanymi algorytmami partycjonowania (\ref{fig:quick-sort-deterministic-pivot-all}) można stwierdzić, że partycjonowanie metodą Hoare wykonuje większą liczbę operacji porównania oraz mniejszą liczbę operacji zamiany miejsc. Czynnik ten może okazać się szczególnie istotny w przypadku sortowania złożonych struktur danych.\\

Do analizy rozkładu prawdopodobieństwa liczby wykonanych operacji (\ref{fig:quick-sort-deterministic-pivot-density}) użyto tablicy losowych danych o stałym rozmiarze $n = 1000$ oraz współczynnika kosztu o stałej wartości $\alpha = 1.0$. Dla badanych algorytmów najmniejsze odchylenie standardowe mają algorytmy partycjonowania metodą Hoare przy wyborze piwota jako pseudo-mediana z dziewięciu lub mediana-median z trzech. Największe odchylenie standardowe występuje dla klasycznego algorytmu Quick Sort. Najmniejsza wartość oczekiwana liczby wykonanych operacji jest osiągana poprzez partycjonowanie metodą Hoare z wyborem piwota jako ostatni element tablicy.

% tabelka, procentowo średnia liczba wykonanych operacji dla każdej z metod przy stałym rozmiarze tablicy n = 1000
% tabelka, wyliczone odchylenie standardowe oraz wartość oczekiwana dla poszczególnych algorytmów
% Wykres porownujacy roznice pomiedzy przypadkiem srednim a pesymistycznym

\begin{figure}[]
	\centering
	\includesvg[inkscapelatex=false,width=0.8\columnwidth]{img/plot/quick-sort-deterministic-pivot-random.svg}
	\caption[]{}
	\label{fig:quick-sort-deterministic-pivot-random}
\end{figure}

\begin{figure}[]
	\centering
	\includesvg[inkscapelatex=false,width=0.8\columnwidth]{img/plot/quick-sort-deterministic-pivot-reversed.svg}
	\caption[]{}
	\label{fig:quick-sort-deterministic-pivot-reversed}
\end{figure}

\begin{figure}[]
	\centering
	\includesvg[inkscapelatex=false,width=0.8\columnwidth]{img/plot/quick-sort-deterministic-pivot-cost-factor.svg}
	\caption[]{}
	\label{fig:quick-sort-deterministic-pivot-cost-factor}
\end{figure}

\begin{figure}[]
	\centering
	\includesvg[inkscapelatex=false,width=0.8\columnwidth]{img/plot/quick-sort-deterministic-pivot-density.svg}
	\caption[]{}
	\label{fig:quick-sort-deterministic-pivot-density}
\end{figure}

\begin{figure}[]
	\centering
	\includesvg[inkscapelatex=false,width=1.1\columnwidth]{img/plot/quick-sort-deterministic-pivot-random-all.svg}
	\caption[]{}
	\label{fig:quick-sort-deterministic-pivot-all}
\end{figure}

\subsection{Wnioski}
W przypadku danych wejściowych typu prostego, tzn. takich, dla których operacja porównania jest czasowo równoważna operacji przypisania, partycjonowanie metodą Hoare okazało się wydajniejsze niż partycjonowanie metodą Lomuto. W przypadku złożonych struktur danych, wydajniejsze jest partycjonowanie metodą Lomuto. Wartość współczynnika kosztu, od której partycjonowanie metodą Lomuto jest bardziej wydajne dla dowolnej strategii wyboru piwota wynosi $\alpha = 2.0$, co można interpretować jako sortowania struktur danych złożonych z co najmniej dwóch typów prostych.\\

Analizując zachowanie algorytmów można stwierdzić, że dla danych posortowanych lub prawie posortowanych, dobrym wyborem jest skorzystanie z kosztownej metody wyszukiwania piwota, która pomimo swojego dodatkowego nakładu czasowego zwiększa prawdopodobieństwo na znalezienie dobrego piwota. Wyjątkiem jest wybór piwota jako mediana-median z pięciu, w przypadku którego dodatkowy nakład czasowy sprawia, że algorytm jest znacznie mniej wydajny nawet dla uporządkowanych danych wejściowych. Przez pojęcie \BOLD{dobrego piwota} rozumiemy tutaj pozycję możliwie blisko środka partycjonowanej tablicy.\\
