\chapter{Podsumowanie}
\thispagestyle{chapterBeginStyle}

Badania były przeprowadzane z podziałem na 4 kategorie. W każdej kategorii został wyłoniony najbardziej optymalny z badanych algorytmów, czyli algorytm wykonujący najmniejszą liczbę operacji, z uwzględnieniem współczynnika kosztu.
Podsumowanie najlepszych algorytmów działających w miejscu
Podsumowanie wyników testowania algorytmów. Wnioski z analizy algorytmów hybrydowych.
Wybrac najlepsze z badanych metod sortujacych z podziałem na 4 kategorie:

- dane losowe, typ podstawowy
Intro Sort, Hoare partition, pseudo-mediana z dziewięciu -> podobna wartość oczekiwana liczby wykonanych operacji do Intro Sort, Hoare partition, mediana z trzech, jednak dla psaeudo-mediany z dziewieciu jest mniejsze odchylenie standardowe

- dane losowe, typ złożony

- dane uporządkowane, typ podstawowy
- dane uporządkowane, typ złożony

Dodać wykres przedstawiający najlepsze algorytmy w każdej z kategorii, w porównaniu do klasycznych Quick Sort oraz merge sort.
