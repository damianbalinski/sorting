\chapter{Podsumowanie}
\thispagestyle{chapterBeginStyle}
W trakcie analizy, eksperymentalnie badano hybrydowe algorytmy sortujące w miejscu, z wykorzystaniem różnych metod partycjonowania oraz deterministycznych i randomizowanych strategii wyboru piwota. Przeprowadzone eksperymenty można podzielić na cztery kategorie, w zależności od danych wejściowych:

\begin{itemize}
	\setlength\itemsep{0em}
	\item losowe dane typu podstawowego,
	\item uporządkowane dane typu podstawowego,
	\item losowe dane typu złożonego,
	\item uporządkowane dane typu złożonego.
\end{itemize}
W każdej z powyższych kategorii wyłoniony został kandydat na najbardziej wydajny algorytm sortujący. Dokonując wyboru sugerowano się łączną liczbą oraz kosztem wykonywanych operacji, z uwzględnieniem wartości współczynnika kosztu.\\

\newpage
\section{Losowe dane typu podstawowego}
Przeprowadzona analiza sugeruje, że najwydajniejszym algorytmem sortującym losowe dane typu podstawowego jest połączenie algorytmu Intro Sort z partycjonowaniem metodą Hoare oraz wyborem piwota jako pseudo-mediana z dziewięciu elementów, przy czym wersja randomizowana wykonuje zbliżoną liczbę operacji.\\

\begin{figure}[H]
	\centering
	\captionsetup{singlelinecheck=off}
	\includesvg[inkscapelatex=false,width=0.8\columnwidth]{img/plot/best-sorting-random-simple-data.svg}
	\caption[]{
		\begin{itemize}
			\item \BOLD{Intro Sort A} - Hoare, pseudo-mediana z dziewięciu
			\item \BOLD{Intro Sort B} - Hoare, pseudo-mediana z dziewięciu wyborów
			\item \BOLD{QuickMerge Sort A} - Hoare, pseudo-mediana z dziewięciu
			\item \BOLD{QuickMerge Sort B} - Hoare, pseudo-mediana z dziewięciu wyborów
			\item \BOLD{Quick Sort A} - Hoare, mediana z trzech
			\item \BOLD{Quick Sort B} - Hoare, losowy elementów
	\end{itemize}}
\end{figure}

\newpage
\section{Uporządkowane dane typu podstawowego}
Analiza eksperymentalna sugeruje, że najwydajniejszym spośród badanych algorytmów, sortującym uporządkowane dane typu podstawowego, jest połączenie algorytmu Intro Sort z partycjonowaniem metodą Hoare oraz wyborem piwota jako pseudo-mediana z dziewięciu elementów tablicy.\\

\begin{figure}[H]
	\centering
	\captionsetup{singlelinecheck=off}
	\includesvg[inkscapelatex=false,width=0.8\columnwidth]{img/plot/best-sorting-sorted-simple-data.svg}
	\caption[]{
		\begin{itemize}
			\item \BOLD{Intro Sort A} - Hoare, pseudo-mediana z dziewięciu
			\item \BOLD{Intro Sort B} - Hoare, pseudo-mediana z dziewięciu wyborów
			\item \BOLD{QuickMerge Sort A} - Hoare, pseudo-mediana z dziewięciu
			\item \BOLD{QuickMerge Sort B} - Hoare, pseudo-mediana z dziewięciu wyborów
			\item \BOLD{Quick Sort A} - Hoare, mediana z trzech
			\item \BOLD{Quick Sort B} - Hoare, losowy elementów
	\end{itemize}}
\end{figure}

\newpage
\section{Losowe dane typu złożonego}
Przeprowadzona analiza sugeruje, że najwydajniejszym algorytmem sortującym losowe dane typu złożonego jest połączenie algorytmu QuickMergeIntro Sort z partycjonowaniem metodą Lomuto oraz wyborem piwota jako pseudo-mediana z dziewięciu losowych elementów. W tym przypadku zastosowanie odpowiednika w postaci QuickMerge Sort prowadzi do podobnych rezultatów.\\

\begin{figure}[H]
	\centering
	\captionsetup{singlelinecheck=off}
	\includesvg[inkscapelatex=false,width=0.8\columnwidth]{img/plot/best-sorting-random-complex-data.svg}
	\caption[]{
		\begin{itemize}
			\item \BOLD{Intro Sort A} - Lomuto, pseudo-mediana z dziewięciu
			\item \BOLD{Intro Sort B} - Lomuto, pseudo-mediana z dziewięciu wyborów
			\item \BOLD{QuickMerge Sort A} - Lomuto, pseudo-mediana z dziewięciu
			\item \BOLD{QuickMerge Sort B} - Lomuto, pseudo-mediana z dziewięciu wyborów
			\item \BOLD{QuickMergeIntro Sort A} - Lomuto, mediana z trzech
			\item \BOLD{QuickMergeIntro Sort B} - Lomuto, pseudo-mediana z dziewięciu wyborów
	\end{itemize}}
\end{figure}

\newpage
\section{Uporządkowane dane typu złożonego}
Analiza eksperymentalna sugeruje, że najwydajniejszym spośród badanych algorytmów, sortującym uporządkowane dane typu złożonego, jest połączenie algorytmu QuickMergeIntro Sort z partycjonowaniem metodą Lomuto oraz wyborem piwota jako mediana z trzech elementów.\\

\begin{figure}[H]
	\centering
	\captionsetup{singlelinecheck=off}
	\includesvg[inkscapelatex=false,width=0.8\columnwidth]{img/plot/best-sorting-sorted-complex-data.svg}
	\caption[]{
		\begin{itemize}
			\item \BOLD{Intro Sort A} - Lomuto, pseudo-mediana z dziewięciu
			\item \BOLD{Intro Sort B} - Lomuto, pseudo-mediana z dziewięciu wyborów
			\item \BOLD{QuickMerge Sort A} - Lomuto, pseudo-mediana z dziewięciu
			\item \BOLD{QuickMerge Sort B} - Lomuto, pseudo-mediana z dziewięciu wyborów
			\item \BOLD{QuickMergeIntro Sort A} - Lomuto, mediana z trzech
			\item \BOLD{QuickMergeIntro Sort B} - Lomuto, pseudo-mediana z dziewięciu wyborów
	\end{itemize}}
\end{figure}
