\chapter{Środowisko uruchomieniowe aplikacji}
\thispagestyle{chapterBeginStyle}
Platforma uruchomieniowa aplikacji - Windows. Wykorzystane języki programowania - C++, Python.

\section{Zmienne środowiskowe}
Opis zmiennych środowiskowych TEST-DIRECTORY, CONFIG-DIRECTORY, PLOT-DIRECTORY.

% potrzebne sa trzy zmienne środowiskowe
% FILE_PATH_CONFIG
% FILE_PATH_TEST
% FILE_PATH_PLOT

\section{Biblioteki zewnętrzne}
Biblioteki w C++ oraz Pythonie potrzebne do uruchomienia aplikacji wraz z numerami wersji.
% C++: fmt - formatowane wyjście z interpolacją napisów

\section{Instalowanie aplikacji}
Uruchamianie skryptu zaciągającego potrzebne zależności.
Uruchamianie pliku makefile kompilującego i instalującego aplikację.
Cykl pracy programu - tworzenie konfiguracji, testowanie silnikiem testującym, wizualizacja wyników przy użycia
silnika graficznego.

\section{Przykładowy plik konfiguracyjny}
Plik konfiguracyjny w jsonie. Omówienie pliku, na początku są dane współdzielone przez wszystkie testy.
Potem plik zawiera listę elementów typu plot, czyli listę osobnych testów.

% TODO screendhot z silnika testujacego
