\newpage
\section{QuickMergeIntro Sort}
- w poprzednich rozdziałach omówione zostały algorytmy QuickMerge Sort oraz Intro Sort, które optymalizują.
W tym rozdziale podjęto próbę połączenia tych algorytmów, tworząc algorytm QuickMergeIntro Sort. Celem tego połączenia jest optymalizacja algorytmu dla przypadku pesymistycznego w klasycznym Quick Sort
- proba polaczenia intro sort z algorytmem quickMerge sort
- stosowanie merge sort ma sens tylko dla zlozonych struktur danych, dlatego wszystkie testy zaklasaja ze $\alpha \geq 10.0$. stosowano tylko partycjonowanie metodą Lomuto, które lepiej sobie radzi dla zlozonych struktur danych.
- biorąc pod uwagę porzednie testy, w badaniach zrezygnowano ze strategii wyboru piwota jako mediana-median, dla których wyniki w porzednich testach były niesatysfakcjonujace

\subsection{Pseudokod}

\begin{algorithm}[H]
	\unboldmath
	\thinmuskip=6mu
	\caption{QuickMergeIntro Sort}
	\begin{algorithmic}[1]
		\Procedure{QuickMergeIntroSort(arr, depth)}{}
		
		\State
		\If {$\FUNC{len}{\VAR{arr}} \leq \VAR{maxLength}$} \Comment{sortowanie przez wstawianie} \label{line:qmi-insertion-sort}
		\State $\CALL{InsertionSort}{\VAR{arr}}$
		
		\State
		\ElsIf {$\VAR{depth} = 0$} \Comment{sortowanie przez kopcowanie} \label{line:qmi-heap-sort}
		\State $\CALL{HeapSort}{\VAR{arr}}$
		
		\State
		\Else \label{line:quick-merge-intro-sort}
		\State $\VAR{arr}_1, \VAR{arr}_2 \gets \CALL{Partition}{\VAR{arr}}$ \Comment{partycjonowanie}
		
		\State
		\If {$\FUNC{len}{\VAR{arr}_1} < \FUNC{len}{\VAR{arr}_2}$} \Comment{sortowanie}
		\State $\VAR{buffer} \gets \VAR{arr}_2$
		\State $\CALL{MergeSortBySwaps}{\VAR{arr}_1, \VAR{buffer}}$
		\State $\CALL{QuickMergeIntroSort}{\VAR{arr}_2, \VAR{depth-1}}$
		\Else
		\State $\VAR{buffer} \gets \VAR{arr}_1$
		\State $\CALL{MergeSortBySwaps}{\VAR{arr}_2, \VAR{buffer}}$
		\State $\CALL{QuickMergeIntroSort}{\VAR{arr}_1, \VAR{depth-1}}$
		\EndIf
		\EndIf
		
		\State
		\EndProcedure
	\end{algorithmic}
	\boldmath
	\thinmuskip=3mu
	\label{pseudocode:quick-merge-intro-sort}
\end{algorithm}


\subsection{Analiza deterministycznych wersji algorytmu QuickMergeIntro Sort}

GESTOSC
- porownywano do najwyjadniejszych odpowiednikow Intro Sort oraz QuickMerge Sort. napisac jakie to odpowiedniki!
- w przypadku danych losowych sa to Intro Sort oraz QuickMergeSort z partycjoinowaniem metodę Lomuto oraz wyborem piwota jako mediana z trzech ...

GESTOSC ALL
- w histogramie pominieto peracje przypisania, dla wiekszosci algorytmow jest ona zerowa a wiec nie wplywa na wydajnosc algorytmow

\begin{figure}[]
	\centering
	\includesvg[inkscapelatex=false,width=1.0\columnwidth]{img/plot/quick-merge-intro-sort-deterministic-pivot-random-sorted.svg}
	\caption[]{}
	\label{fig:quick-merge-intro-sort-deterministic-pivot-random-sorted}
\end{figure}

\begin{figure}[]
	\centering
	\includesvg[inkscapelatex=false,width=1.0\columnwidth]{img/plot/quick-merge-intro-sort-deterministic-pivot-density.svg}
	\caption[]{}
	\label{fig:quick-merge-intro-sort-deterministic-pivot-density}
\end{figure}

\begin{figure}[]
	\centering
	\includesvg[inkscapelatex=false,width=1.0\columnwidth]{img/plot/quick-merge-intro-sort-deterministic-pivot-density-sorted.svg}
	\caption[]{}
	\label{fig:quick-merge-intro-sort-deterministic-pivot-density-sorted}
\end{figure}

\subsection{Analiza randomizowanych wersji algorytmu QuickMergeIntro Sort}

GESTOSC
- porownywano do najwyjadniejszych odpowiednikow Intro Sort oraz QuickMerge Sort. napisac jakie to odpowiedniki!
- w przypadku danych losowych sa to Intro Sort oraz QuickMergeSort z partycjoinowaniem metodę Lomuto oraz wyborem piwota jako mediana z trzech losowych elementow talbvcy 

\begin{figure}[]
	\centering
	\includesvg[inkscapelatex=false,width=1.0\columnwidth]{img/plot/quick-merge-intro-sort-nondeterministic-pivot-random-sorted.svg}
	\caption[]{}
	\label{fig:quick-merge-intro-sort-nondeterministic-pivot-random-sorted}
\end{figure}

\begin{figure}[]
	\centering
	\includesvg[inkscapelatex=false,width=1.0\columnwidth]{img/plot/quick-merge-intro-sort-nondeterministic-pivot-density.svg}
	\caption[]{}
	\label{fig:quick-merge-intro-sort-nondeterministic-pivot-density}
\end{figure}

\begin{figure}[]
	\centering
	\includesvg[inkscapelatex=false,width=1.0\columnwidth]{img/plot/quick-merge-intro-sort-nondeterministic-pivot-density-sorted.svg}
	\caption[]{}
	\label{fig:quick-merge-intro-sort-nondeterministic-pivot-density-sorted}
\end{figure}

\subsection{Wnioski}
- bardzo podobny do QuickMerge Sort, gorszy dla losowych danych, lepszy dla uporzadkowanych danych, dlatego bedzie preferowanym algorytmem dzialajacym w miejscu