\chapter{Słownik pojęć}

\section{Notacja duże-$O$}
Określa asymptotyczne tempo wzrostu danej funkcji. Wykorzystywana jest do opisu złożoności obliczeniowej.
Mówimy, że funkcja $f$ ma złożoność rzędu $g$, tzn. $f \in O(g)$, jeśli spełniony jest warunek:
$$(\exists n_0 > 0) (\exists c > 0) (\forall n \geq n_0) f(n) \leq c \cdot g(n)$$

\section{Algorytm działający w miejscu}
Aby efektywnie posortować tablicę wejściową algorytm potrzebuje co najwyżej $O(\log{n})$ dodatkowej pamięci. Algorytmem działającym w miejscu nie jest klasyczny Merge Sort.\\

\section{Algorytm stabilny}
Algorytm stabilny w trakcie sortowania nie zamienia kolejnością elementów równoważnych względem porządku sortowania. Relatywne ułożenie elementów o tej samej wartości pozostaje takie samo.\\