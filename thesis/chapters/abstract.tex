\begin{streszczenie}
Praca składa się z pięciu rozdziałów.
Rozdział pierwszy poświęcony jest metodologii prowadzenia badań. W rozdziale tym wyjaśniono sposób, w jaki zliczano liczbę wykonywanych operacji porównania, przypisania oraz zamiany miejsc, uwzględniając koszt każdej operacji.
Rozdział drugi zawiera przegląd podstawowych algorytmów sortujących, z wyszczególnieniem problemów, jakie wiążą się ze stosowania tych algorytmów.
W rozdziale trzecim przedstawiono nowoczesne sposoby na tworzenie algorytmów hybrydowych. W rozdziale tym dokonano analizy porównawczej opisywanych algorytmów, z podziałem na wersje deterministyczne oraz randomizowane.
Rozdział czwarty zawiera podsumowanie wyników z wcześniej przeprowadzonych badań. W rozdziale tym wyłoniono kandydatów na najbardziej optymalny algorytm w danej kategorii.
Ostatni rozdział to opis implementacji oraz wykorzystanych technik programowania. W tym rozdziale opisano koncepcję silnika testującego oraz wzorce projektowe użyte w systemie.\\
\end{streszczenie}

\vspace*{1cm}
\begin{abstract}
The engineering thesis consists of five chapters.
The first chapter describes the research methodology. That chapter explains how the number of performed operations was counted, broken down into compare, assign and swap operation, including the cost of each operation.
The second chapter provides an overview of the basic sorting algorithms and problems associated with these algorithms.
The third chapter presents modern methods of creating hybrid sorting algorithms. That chapter presents a comparative analysis of described algorithms, divided into deterministic and randomized versions.
Chapter four summarizes the results of the previous research. That chapter describes the candidate for the most optimal sorting algorithm in a given category.
The last chapter describes the implementation and programming techniques. That chapter describes the concept of test engine and the design patterns used in the system.\\
\end{abstract}
