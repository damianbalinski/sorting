\chapter{Przegląd hybrydowych algorytmów sortujących}
\thispagestyle{chapterBeginStyle}

\section{Główne sposoby modyfikacji algorytmów}
W celu poprawy wydajności algorytmów sortujących stosuje się ich modyfikacje oraz ulepszenia. W tej pracy wykorzystano dwa główne sposoby na usprawnienie algorytmów.\\

Pierwszym sposobem jest modyfikacja składowych danego algorytmu. Wiele spośród znanych algorytmów składa się z kilku osobnych kroków, z których każdy można wyekstrahować do oddzielnego procesu. Pomysł ten polega na modyfikacji składowych algorytmu sortującego w taki sposób, aby np. lepiej radził on sobie w przypadku pesymistycznym.\\

Drugim sposobem na ulepszenie jest próba połączenia wielu algorytmów sortujących. Niektóre algorytmy zachowują się lepiej dla stosunkowo małej ilości danych, inne zaś są znaczenie wydajniejsze przy rozbudowanym zbiorze danych wejściowych. Pomysł ten polega na opracowaniu algorytmu, którego działanie zmienia się w zależności od czynników zewnętrznych, np. długości danych do posortowania.\\

\section{Rodzina algorytmów Quick Sort z deterministycznym algorytmem wyboru pivota}
% Algorytm median-of-medians gwarantuje wybor dobrego pivota przy dodatkowym nakladzie czasowym, ktory redukuje sie dla 
% przypadku pesymistycznego.
Algorytmy Quick Sort z różnymi algorytmami partycjonowania oraz różnymi algorytmami wyboru pivota
Wykresy porównujące algorytmy.\\
1. Partycjonowanie metodą Lemuto (domyślne)\\
2. Partycjonowanie metodą Hoare\\
3. Wybór pivota metodą median of three - pierwszy, środkowy, ostatni.\\
4. Wybór pivota jako mediana of medians\\
5. Wybór pivota metodą pseudomedian of nine\\
6. Wybór pivota algorytmem quick select

\begin{figure}[H]
	\centering
	\includesvg[inkscapelatex=false,width=0.8\columnwidth]{img/plot/quick-sort-deterministic-pivot-random.svg}
	\caption[]{}
	\label{fig:quick-sort-deterministic-pivot-random}
\end{figure}

\begin{figure}[H]
	\centering
	\includesvg[inkscapelatex=false,width=0.8\columnwidth]{img/plot/quick-sort-deterministic-pivot-reversed.svg}
	\caption[]{}
	\label{fig:quick-sort-deterministic-pivot-reversed}
\end{figure}

\section{Rodzina algorytmów Quick Sort z niedeterministycznym algorytmem wyboru pivota}
1. QuickSort z losowaniem pivota\\
2. QuickSort z losowaniem trzech liczb, wybór mediany (power of three choices)\\

\section{QuickMerge Sort}
Koncepcja algorytmu, mocne strony (Merge Sort bez konieczności alokacji pamięci)

\subsection{Pseudokod}

\subsection{Analiza algorytmu}
Wykresy liczby wykonywanych operacji w porównaniu do algorytmów bazowych.
Wykresy gęstości liczby wykonywanych operacji dla stałej liczby n, np n = 10000.

\subsection{Wnioski}
Wyniki analizy porównawczej


\section{Intro Sort}
Ogólny opis algorytmu, gdzie jest wykorzystywany (std::sort w g++), zalety.
% TODO sprawdzic jakiego algorytmu partycjonowania uzywaja w srodku i spróbować podmienić, zmienic na wariacje z randomowym
% wyborem pivota
\subsection{Pseudokod}

\subsection{Analiza algorytmu}
Wykresy liczby wykonywanych operacji w porównaniu do algorytmów bazowych.
Wykresy gęstości liczby wykonywanych operacji dla stałej liczby n, np n = 10000.
Wykresy dla różnych algorytmów partycjonowania.

\subsection{Wnioski}
Wyniki analizy porównawczej

% TODO jak zdaze proba modyfikacji algorytmu intro sort

% TODO jak zdaze dodac Quick Heap Sort

% TODO merge sort na wiele czesci
