\chapter{Przegląd podstawowych algorytmów sortujących}
\thispagestyle{chapterBeginStyle}


\section{Quick Sort}
Ogólna koncepcja algorytmu, autor, rok powstania, bez pseudokodu, gdzie algorytm jest wykorzystywany
(java)

% Algorytm ten oraz jego wariacje są nadal powszechne w wielu systemach komupetorych [link do java8] 
% Dziel i zwyciężaj, partycjonowanie na dwie podtablice, i sortowanie kazdej oddzielnie
% Tabelska sortowania
% British computer scientist Tony Hoare in 1959
% algorytm yen odcisna pietno w historii, na jego czesc metoda sortujaca w systremach unix naztwa sie qsort

\subsection{Analiza algorytmu Quick Sort}
Wykresy liczby operacji porównania, zamiany miejsc, przypisania. Eksperymentalna analiza wartości
oczekiwanej liczby operacji.

\begin{figure}[H]
	\centering
	\includesvg[inkscapelatex=false,width=1.0\columnwidth]{img/plot/quick-sort.svg}
\end{figure}

\subsection{Problemy związane z algorytmem Quick Sort}
Wiele kosztownych operacji porównania, przeciwieństwo algorytmu Merge Sort.
Słaba pesymistyczna złożoność czasowa.

\section{Merge Sort}
Ogólna koncepcja algorytmu, autor, rok powstania, bez pseudokodu.

\subsection{Analiza algorytmu Merge Sort}
Wykresy liczby operacji porównania, zamiany miejsc, przypisania. Eksperymentalna analiza wartości
oczekiwanej liczby operacji.

% Tabelka sortowania

\subsection{Problemy związane z algorytmem Merge Sort}
Konieczność alokacji dodatkowych zasobów.