\begin{algorithm}[H]
	\unboldmath
	\thinmuskip=6mu
	\caption{QuickMerge Sort}
	\begin{algorithmic}[1]
		\Procedure{QuickMergeSort(arr, n)}{}
		
		\State
		\If {$n = 1$} \COMMAND{end} \Comment{warunek wyjścia}
		\EndIf
		
		\State
		\State $\VAR{pivot}, n_1, n_2 \gets \CALL{Partition}{\VAR{arr}, n}$ \Comment{partycjonowanie}
		
		\State
		\If {$n_1 < n_2$} \Comment{sortowanie}
			\State $\VAR{buffer} \gets \VAR{arr+pivot}$
			\State $\CALL{MergeSortBySwaps}{\VAR{arr}, n_1, \VAR{buffer}}$
			\State $\CALL{QuickMergeSort}{\VAR{arr+pivot}, n_2}$
		\Else
			\State $\VAR{buffer} \gets \VAR{arr}$
			\State $\CALL{MergeSortBySwaps}{\VAR{arr+pivot}, n_2, \VAR{buffer}}$
			\State $\CALL{QuickMergeSort}{\VAR{arr}, n_1}$
		\EndIf
		
		\EndProcedure
	\end{algorithmic}
	\boldmath
	\thinmuskip=3mu
	\label{pseudocode:quick-merge-sort}
\end{algorithm}
