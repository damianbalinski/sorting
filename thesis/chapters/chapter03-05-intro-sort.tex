\section{Intro Sort}
Ogólny opis algorytmu, gdzie jest wykorzystywany (std::sort w g++), zalety.
% TODO sprawdzic jakiego algorytmu partycjonowania uzywaja w srodku i spróbować podmienić, zmienic na wariacje z randomowym
% wyborem pivota
% SORTOWANIE INTROSPEKTYWNE 

\subsection{Pseudokod}

\begin{algorithm}[H]
	\unboldmath
	\thinmuskip=6mu
	\caption{Intro Sort}
	\begin{algorithmic}[1]
		\Procedure{IntroSort(arr, depth)}{}
		
		\State
		\If {$\FUNC{len}{\VAR{arr}} < \VAR{maxLength}$} \Comment{sortowanie przez wstawianie}
			\State $\CALL{InsertionSort}{\VAR{arr}}$
		
		\State
		\ElsIf {$\VAR{depth} = 0$} \Comment{sortowanie przez kopcowanie}
		\State $\CALL{HeapSort}{\VAR{arr}}$
		
		\State
		\Else \Comment{szybkie sortowanie}
			\State $\VAR{arr}_1, \VAR{arr}_2 \gets \CALL{Partition}{\VAR{arr}}$
			\State $\CALL{IntroSort}{\VAR{arr}_1, \VAR{depth-1}}$
			\State $\CALL{IntroSort}{\VAR{arr}_2, \VAR{depth-1}}$
		\EndIf
		
		\State
		\EndProcedure
	\end{algorithmic}
	\boldmath
	\thinmuskip=3mu
	\label{pseudocode:intro-sort}
\end{algorithm}


\subsection{Analiza deterministycznych wersji algorytmu Intro Sort}
Wykresy liczby wykonywanych operacji w porównaniu do algorytmów bazowych.
Wykresy gęstości liczby wykonywanych operacji dla stałej liczby n, np n = 10000.
Wykresy dla różnych algorytmów partycjonowania.

\begin{figure}[]
	\centering
	\includesvg[inkscapelatex=false,width=1.0\columnwidth]{img/plot/intro-sort-deterministic-pivot-random.svg}
	\caption[]{}
	\label{fig:intro-sort-deterministic-pivot-random}
\end{figure}

\begin{figure}[]
	\centering
	\includesvg[inkscapelatex=false,width=1.0\columnwidth]{img/plot/intro-sort-deterministic-pivot-reversed.svg}
	\caption[]{}
	\label{fig:intro-sort-deterministic-pivot-reversed}
\end{figure}

\begin{figure}[]
	\centering
	\includesvg[inkscapelatex=false,width=1.1\columnwidth]{img/plot/intro-sort-deterministic-pivot-random-all.svg}
	\caption[]{}
	\label{fig:intro-sort-deterministic-pivot-random-all}
\end{figure}

\subsection{Analiza randomizowanych wersji algorytmu Intro Sort}

\subsection{Wnioski}
Wyniki analizy porównawczej
