\section{Intro Sort}
Ogólny opis algorytmu, gdzie jest wykorzystywany (std::sort w g++), zalety.
% TODO sprawdzic jakiego algorytmu partycjonowania uzywaja w srodku i spróbować podmienić, zmienic na wariacje z randomowym
% wyborem pivota
% SORTOWANIE INTROSPEKTYWNE 

\subsection{Pseudokod}

\subsection{Analiza deterministycznych wersji algorytmu Intro Sort}
Wykresy liczby wykonywanych operacji w porównaniu do algorytmów bazowych.
Wykresy gęstości liczby wykonywanych operacji dla stałej liczby n, np n = 10000.
Wykresy dla różnych algorytmów partycjonowania.

\begin{figure}[]
	\centering
	\includesvg[inkscapelatex=false,width=1.0\columnwidth]{img/plot/intro-sort-deterministic-pivot-random.svg}
	\caption[]{}
	\label{fig:intro-sort-deterministic-pivot-random}
\end{figure}

\begin{figure}[]
	\centering
	\includesvg[inkscapelatex=false,width=1.1\columnwidth]{img/plot/intro-sort-deterministic-pivot-random-all.svg}
	\caption[]{}
	\label{fig:intro-sort-deterministic-pivot-random-all}
\end{figure}

\subsection{Analiza randomizowanych wersji algorytmu Intro Sort}

\subsection{Wnioski}
Wyniki analizy porównawczej
