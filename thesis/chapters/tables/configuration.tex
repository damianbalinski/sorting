\begin{table}[H]
	\centering
	\def\arraystretch{1.5}
	\begin{tabular}{ll}
		\multicolumn{2}{c}{}                									\\ \hline
		\BOLD{title}			& tytuł wizualizacji							\\ \hline
		\BOLD{output}			& nazwa pliku wyjściowego						\\ \hline
		\BOLD{prefix}			& prefiks dodawany do nazw plików pośrednich	\\ \hline
		\BOLD{grid}				& układ wykresów na wizualizacji $(x,y)$		\\ \hline
		\BOLD{size}				& rozmiar wizualizacji $(x,y)$					\\ \hline
		\BOLD{invariants}		& rozmiar tablicy oraz liczba powtórzeń			\\ \hline
		\BOLD{range}			& rozmiar tablicy								\\ \hline
		\BOLD{begin}			& początkowy rozmiar tablicy					\\ \hline
		\BOLD{end}				& końcowy rozmiar tablicy						\\ \hline
		\BOLD{step}				& inkrement podczas zmiany rozmiaru tablicy		\\ \hline
		\BOLD{repeats}			& liczba powtórzeń dla każdego z testów			\\ \hline
		\BOLD{plots}			& osobne konfiguracje dla poszczególnych wykresów \\ \hline
		\BOLD{type}				& rodzaj testu 									\\ \hline
		\BOLD{metadata}			& parametry graficzne dla pojedynczego wykresu 	\\ \hline
		\BOLD{title}			& tytuł wykresu 								\\ \hline
		\BOLD{xlabel}			& podpis osi odciętych 							\\ \hline
		\BOLD{ylabel}			& podpis osi rzędnych 							\\ \hline
		\BOLD{xcolumn}			& dane do osi odciętych 						\\ \hline
		\BOLD{ycolumn}			& dane do osi rzędnych 							\\ \hline
		\BOLD{xmin}				& minimalna wartość na osi odciętych 			\\ \hline
		\BOLD{xmax}				& maksymalna wartość na osi odciętych 			\\ \hline
		\BOLD{legend}			& pozycja legendy 								\\ \hline
		\BOLD{functions}		& funkcje pomocnicze drukowane na wykresie 		\\ \hline
		\BOLD{soirtings}		& wyniki testów drukowane na wykresie 			\\ \hline
		\BOLD{label}			& podpis linii 									\\ \hline
		\BOLD{color}			& kolor linii 									\\ \hline
		\BOLD{linestyle}		& styl linii 									\\ \hline
		\BOLD{expression}		& generator dla funkcji pomocniczej				\\ \hline
		\BOLD{algorithm}		& algorytm dla przeprowadzanego testu			\\ \hline
		\BOLD{generator}		& generator danych wejściowych przeprowadzanego testu \\ \hline
	\end{tabular}

	\caption[]{Opis parametrów zawartych w pliku konfiguracyjnym.}
	\label{tab:configuration}
\end{table}
