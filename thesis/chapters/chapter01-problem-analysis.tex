\chapter{Analiza problemu}

\section{Model matematyczny przypadku średniego}

\section{Model matematyczny przypadku pesymistycznego}

\section{Założenia}

\subsection{Założenia odnośnie testowanych parametrów}
Testowana będzie złożoność czasowa, w tym celu analizowane są takie parametry jak: liczba porównań,
liczba swapów oraz liczba przypisań, z pominięciem operacji wykonywanych na iteratorach pętli -
wyjaśnienie dlaczego.

\subsection{Założenia odnośnie danych wejściowych}
Wiele algorytmów sortujących bazuje na pewnych założeniach odnośnie wejściowego zbioru danych.
Np. algorytm ... doskonale radzi sobie ze zbiorem danych prawie posortowanym, tzn.
takim w którym ... . W tej pracy zakładamy że dane wejściowe będą losowym ciągiem liczb powtórzeniami.
