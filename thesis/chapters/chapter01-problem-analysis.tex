\chapter{Analiza problemu}

\section{Model matematyczny przypadku średniego}

\section{Model matematyczny przypadku pesymistycznego}

\section{Założenia}

\subsection{Założenia odnośnie testowanych parametrów}
TODO: Testowana będzie złożoność czasowa, w tym celu analizowane są takie parametry jak: liczba porównań,
liczba swapów oraz liczba przypisań, z pominięciem operacji wykonywanych na iteratorach pętli -
wyjaśnienie dlaczego.\\

TODO: Ponieważ rzeczywisty czas wykonywania algorytmu różni się w zależności od maszyny oraz architektury systemu
na którym przeprowadzany test, zostało przyjęte następujące założenie:
Złożoność czasowa została określona wzorem:

TODO silnik testujący zlicza liczbę wykonanych operacji atomowych. Do operacji atomowych zaliczamy: operację porównania, zamiany miejsc oraz przypisania. W badanym systemie pominięte zostały operacje przeprowadzane na indeksach oraz iteratorach. Powodem tej decyzji jest fakt, że zarówno indeksy, jak i iteratory w większości systemów są typami podstawowymi, koszta takich operacji jest niewspółmiernie mniejszy niż np. porównywanie typów złożonych. W tej sytuacji zdecydowano się całkowicie pominąć liczbę wykonywanych operacji na typach prymitywnych.

TODO łączna liczba wykonanych operacji jest sumą ważoną operacji atomowych.
TODO wspolczynnik kosztu \boldmath$\alpha$, czyli ile razy kosztowniejsze jest porownanie od przypisania. zaklasamy ze zmiana miejsc to trzy przypisania. Dla potrzeb testow podstawowych przyjmuje sie \boldmath$\alpha = 1.0$.
TODO w prostej analizie dla typow podstawowych przyjmujemy ze operacja zamiany miejsc jest 3 razy bardziej kosztowna od operacji przypisania oraz porownania.
\boldmath$$T = n_c + 3 \cdot n_s + n_a $$

TODO w celu przeprowadzenia analizy porównawczej wprowadzono pojęcie współczynnika kosztu.

Gdzie:\\
\boldmath$T$ - czas trwania algorytmu\\
$n_c$ - liczba operacji porównania\\
$n_s$ - liczba operacji zamiany miejsc\\
$n_a$ - liczba operacji przypisania\\

Ponieważ wszystkie algorytmy zawarte w silniku testującym są rekurencyjne, podczas badania algorytmów pominięto koszt związany z rekurencyjnym przejściem drzewa wywołania.
Ponieważ w analizie porównawczej uwzględniono tylko algorytmy działające w miejscu, podczas badania nie uwzględniano kosztu alokacji pamięci, która dla każdego algorytmu miejscowego jest rzędu $O(1)$.

Rozróżnić łączny koszt operacji od łącznej liczby operacji.

\subsection{Założenia odnośnie danych wejściowych}
TODO: Wiele algorytmów sortujących bazuje na pewnych założeniach odnośnie wejściowego zbioru danych.
Np. algorytm ... doskonale radzi sobie ze zbiorem danych prawie posortowanym, tzn.
takim w którym ... . W tej pracy zakładamy że dane wejściowe będą losowym ciągiem liczb powtórzeniami.
